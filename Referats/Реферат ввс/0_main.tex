\documentclass[referat, times]{SCWorks}

\usepackage{preamble}

\begin{document}

% Кафедра (в родительном падеже)
\chair{информатики и программирования}

% Тема работы
\title{РАЗРАБОТЧИК ВЕБ-ПРИЛОЖЕНИЙ}

% Курс
\course{1}

% Группа
\group{111}

% Специальность/направление код - наименование
\napravlenie{02.03.02 Фундаментальная информатика и информационные технологии}

% Фамилия, имя, отчество в родительном падеже
\author{Цой Игоря Валерьевича}

% Год выполнения отчета
\date{2024}

\maketitle

\tableofcontents

\intro

Разработка веб-приложений является одной из ключевых областей информационных технологий в современном цифровом мире. Веб-приложения играют важную роль в обеспечении пользователей доступом к различным сервисам, от социальных сетей и онлайн-магазинов до банковских услуг и развлекательных платформ. Разработчики веб-приложений являются основными исполнителями и инноваторами в этой области, поскольку они отвечают за создание и поддержание веб-приложений, обеспечивая их функциональность, производительность, безопасность и удобство использования. В данном реферате будет рассмотрено важное значение профессии разработчика веб-приложений, их обязанности и  требуемые навыки.

% здесь \section, \subsection ...

% 1 глава
\section{\textbf{ВИДЫ ВЕБ-РАЗРАБОТЧИКОВ}}
    Разработчик --- широкий термин для группы специалистов, работа которых направлена на создание мобильных и компьютерных приложений, игр, баз данных и прочего программного обеспечения самых различных устройств \cite{1}. Разработчики в своей деятельности умело совмещают творческий подход и строгий язык программирования. Разработчик включает в себя более глубокое понятие, чем программист. Работа программистов заключается в четком написании алгоритмов по уже готовому запросу. В то время как разработчики рассматривают проект с нуля, определяя цель, идею, тематику и прочие важные элементы. Другими словами, обязанности разработчиков гораздо обширнее. Если программист выполняет функции исполнителя, то разработчик занимается проектом в целом \cite{2}.

\subsection{Frontend-разработчик}
Фронтенд --- клиентская сторона пользовательского интерфейса к
программно-аппаратной части сервиса.
Пользовательский интерфейс --- интерфейс, обеспечивающий передачу
информации между пользователем-человеком и программно-аппаратными
компонентами компьютерной системы.
Благодаря высокому темпу развития технологий, сайтостроение
занимает немаловажную нишу в области сферы программирования.
Непосредственно, при создании веб-сайта участвует верстальщик.
Верстальщик --- это специалист, задачей которого является корректное и
единообразное отображение веб-сайта во всех браузерах и на всех
электронных устройствах (ПК, ноутбук, смартфон, планшет)\cite{3}. 

\subsection{Backend-разработчик}
Backend разработчики занимаются разработкой, исправлением и
изменением аспектов программного приложения или информационной системы, которые
обычный пользователь никогда не видит. Они создают основные функции и компоненты
программ, которые определяют внутреннее содержимое и функционал системы.
Когда пользователь задает запрос системе через внешний интерфейс (Front End), то
Backend-разработчик должен убедиться в том, что программа может решить все
поставленные. Backend-разработчики имеют и другие обязанности: обслуживание
центральных баз данных, управление интерфейсами прикладных программ (API), а также
тестирование и отладка серверных процессов, т.е. полная гарантия эффективного
функционирования программы или приложения\cite{4}.

\subsection{Fullstack-разработчик}
Fullstack-разработка --- это процесс создания веб-приложений, который включает в себя
создание и frontend, и backend. Frontend --- это пользовательский интерфейс приложения,
который пользователь видит и взаимодействует с ним. Backend --- это часть приложения, которая
выполняет бизнес-логику, обрабатывает запросы от frontend и связывается с базой данных. В
Fullstack-разработке разработчик занимается созданием как frontend, так и backend, что
позволяет ему создавать полноценные веб-приложения от начала до конца. Fullstackразработка --- это сложный и многогранный процесс, который требует от разработчика широкого
кругозора и глубоких знаний в различных областях программирования. Однако, благодаря
этому подходу, можно создавать более сложные и функциональные приложения, которые
могут удовлетворить потребности самых требовательных пользователей \cite{5}.

%2 глава
\section{\textbf{ТРЕБОВАНИЕ К НАВЫКАМ}}
 \subsection{Знание языков программирования}

JavaScript — это язык программирования, который даёт возможность реализовывать сложное поведение веб-страницы. Каждый раз, когда вы видите веб-страницу, она не только отображает статическое содержимое, но и делает большее --- своевременно отображает обновление контента, выводит интерактивные карты, 2D/3D анимацию, прокручивает видео и т.д. --- будьте уверены, здесь не обошлось без JavaScript\cite{6}.

Python --- мультипарадигмальный язык программирования: он позволяет совмещать процедурный подход к написанию кода с объектно-ориентированным и функциональным.  Элегантный дизайн и эффективный, дисциплинирующий синтаксис этого языка облегчают программистам совместную работу над кодом.

Ruby --- динамический императивный объектно-ориентированный язык программирования, разработанный Юкихиро Матсумото. Ruby был создан под влиянием таких языков, как Perl, Eiffel и Smalltalk.

PHP (Hypertext PreProcessor, препроцессор гипертекста) --- язык программирования, исполняемый на стороне веб-сервера, спроектированный Расмусом Лердорфом (Rasmus Lerdorf) в качестве инструмента создания динамических и интерактивных веб-сайтов\cite{7}.

\subsection{Умение работать с фреймворками}

Фреймворк (англ. Framework — «каркас», «структура») --- это динамически пополняемая библиотека языка программирования, в которой собраны его базовые модули. Фреймворки создаются для упрощения процессов разработки приложений, сайтов, сервисов. Чтобы не писать модуль в приложении с нуля, гораздо проще обратиться к готовым шаблонам фреймворков, которые и формируют рабочую среду разработчика.

ReactJS --- это библиотека JavaScript, созданная Facebook в 2013 году, она превосходно подходит для создания масштабных веб-приложений, где данные могут меняться на регулярной основе.

Angular ---  фреймворк с открытым исходным кодом, разработанный и поддерживаемый Google. Инструмент дает все необходимое для создания и управления динамическими front-end страницами для веб-приложения.

Django --- высокоуровневый фреймворк, который является не только быстрым решением в веб-разработке, включающим все необходимое для качественного кода и прозрачного написания, но также и отличной платформой для работы с клиентурой того или иного бизнеса. Вместе с тем он удобен для разработчиков\cite{8}.

\subsection {Опыт работы с базами данных}

Все современные автоматизированные системы используют для хранения данных либо хранилища, либо базы данных. Существует множество видов систем управления базами данных, поддерживающих реляционные, объектные, документно-ориентированные и другие модели данных.

SQL --- это мощная и надежная система управления данными, обеспечивающая множество функций, защиту данных и высокую производительность для внедренных приложений-клиентов, «легких» веб-приложений и локальных хранилищ данных.

MongoDB --- документо-ориентированная СУБД с открытым исходным кодом, не требующая описания схемы таблиц\cite{9}.


%3 глава    
\section{\textbf{ТЕНДЕНЦИИ В ВЕБ-РАЗРАБОТКЕ}}
\subsection{Использование Progressive Web Applications (PWA)}
PWA --- новый тип приложений, позволяющий создавать богатый пользовательский интерфейс, работать в автономном режиме и не требовать установки на устройство. Разработка PWA является одним из наиболее эффективных способов создания веб-приложений, которые могут быть запущены на различных устройствах и иметь функциональность, сопоставимую с нативными приложениями\cite{10}.

\subsection{Развитие технологий для создания мобильных веб-приложений}
Разработка под iOS происходит в среде разработки XCode на языке Swift (а раньше --- на Objective-C).
При использовании технологии разработки мобильных приложений на платформе андроид используется среда Android Studio и язык Kotlin (до 2018 года основным языком был Java)\cite{11}.

\subsection{Внедрение искусственного интеллекта и машинного обучения в веб-приложения}
Искусственный интеллект (ИИ) и машинное обучение (МО) играют все более важную роль в современной веб-разработке. Использование ИИ и МО позволяет ускорить и улучшить процесс разработки, оптимизировать работу и повысить качество готового продукта\cite{12}.


%4 Глава
\section{\textbf{ОСНОВНЫЕ ЭТАПЫ РАЗРАБОТКИ}}
Процесс разработки веб-сайта --- сложный и многогранный процесс,
состоящий из многих этапов. Простыми словами это список шагов, которые вы
должны предпринять от начала до конца, чтобы завершить проект.
В более широком смысле веб-разработка включает в себя все действия,
обновления и операции, необходимые для создания, обслуживания и управления
веб-сайтом, для обеспечения его оптимальной производительности, удобства для
пользователей и скорости.
Рассмотрим основные этапы разработки веб-сайта.

1. Сбор информации
Выяснение ваших потребностей и знание того, что вы хотите, имеет
решающее значение для получения необходимой информации. Как только вы
узнаете свои требования, устранение вариантов и принятие решений станет
проще.

2. Планирование
Когда у вас уже есть вся необходимая информация, вы должны ее
спланировать. Структурированное планирование веб-сайта задаст четкое
направление вашей деятельности, обеспечит эффективность нового веб-сайта
для бизнеса и удобство для пользователей. Однако этот процесс не так прост, как
кажется.
У вас должен быть четкий пошаговый план того, что вы ожидаете от этого
веб-сайта. Планирование всегда окупается независимо от того, делаете ли вы его
сами или нанимаете кого-то.

3. Выбор хостинга и доменного имени
Перед началом разработки веб-сайта и создания его дизайна, необходимо
первым делом выбрать хостинг. Он может быть как платным и бесплатным.
Желательно выбрать платный вариант, поскольку во втором имеются
ограничения и для повышенной функциональности некоторые услуги нужно
будет докупать.
Простыми словами хостинг --- это процесс аренды или покупки места для
размещения веб-сайта во всемирной паутине. Содержимое веб-сайта, такое как
контент и изображения, должно быть размещено на сервере, чтобы его можно
было просматривать в Интернете.

4. Разработка дизайн-макета сайта
После того, как вы получили всю необходимую информацию,
спланировали весь процесс и выбрали, где будет «находиться» сайт, необходимо
заняться конструированием его дизайна (или как его называют «фронтендом»).
От того, как ваш сайт будет выглядеть внешне, напрямую зависит то, как
его воспримут клиенты (или просто посетители). Если он плохо структурирован,
бессистемен и/или беспорядочен, весьма вероятно, что вы потеряете клиентов.
Поэтому очень важно уделить особое внимание данному этапу.

5. Создание системной части сайта (вёрстка)
На данном этапе, как и в прошлом, можно либо обратиться к
профессионалу (что желательно, если у вас нет необходимых знаний; ведь ваш 
сайт --- это ваш бизнес, а обращение к специалисту сулит шансы на более высокую
рентабельность и сэкономит время и силы), либо сделать всё самому. Этот
процесс также именуют «бэкендом».

6. Контент сайта, наполнение информацией
Контент --- самый важный элемент веб-сайта. Вы должны инвестировать
свои ресурсы в создание более качественных страниц вместо того, чтобы
добавлять новые страницы, которые вы не можете регулярно поддерживать и
оптимизировать: качественная и грамотно структурированная информация
привлекает клиентов и позволяет завоевать их доверие. Также информацию
нужно своевременно обновлять, дополнять или изменять, если это необходимо.

7. Тестирование, обзор и запуск
После завершения всех этапов вы должны просмотреть и удостовериться,
что всё соответствует вашим ожиданиям.
Не спешите запускать свой веб-сайт, если он не такой, каким, по вашему
мнению, должен быть. Его следует должным образом протестировать, например,
убедиться, что все формы работают, и проверить на наличие ошибок в
содержании и дизайне\cite{13}.

%заключение
\conclusion
Появление на свет веб-приложений и связанных с ними технологий ускорило процесс цифровизации. Государственные инстанции, частные компании, обычные бытовые задачи, все они зачастую решаются благодаря информационным системам на основе веб-приложений или позволяют его в крайнем случае упростить.
Перспективы развития профессии веб-разработчика представляют собой разнообразные и увлекательные возможности в свете растущего влияния цифровизации и технологического прогресса. Они обещают множество увлекательных возможностей и вызовов, связанных с инновациями и развитием технологий, что делает данную профессию важной и актуальной в наши дни.

% Отобразить все источники. Даже те, на которые нет ссылок.
\nocite{*}

% \inputencoding{cp1251}
\bibliographystyle{gost780uv}
\bibliography{thesis}
% \inputencoding{utf8}

% Окончание основного документа и начало приложений
% Каждая последующая секция документа будет являться приложением
\appendix

\end{document}